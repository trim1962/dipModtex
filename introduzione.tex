\chapter{Introduzione}
\section{I mezzi}
\lstset{language=[LaTeX]TeX} 
Che cosa è quello che segue? certo non è un manuale di programmazione, non è um manuale per costruire classi, non è una raccolta dei principali pacchetti \LaTeX ma un elenco di classi di cui vengono indicate le dipendenze.   

\lstinputlisting[caption={Esempio Keyval 1},%
captionpos=b,label=lst:keyvalesempio1,float]{test/Mkeyval1.tex} 

Il listato~\vref{lst:keyvalesempio1} è un esempio minimo di file da compilare. L'esempio è stato costruito per vedere cosa è necessario per la compilazione di \textit{Keyval}. Per mostrare queste dipendenze sono stati utilizzati vari metodi.

La partenza per il tutto  è   la lettura del file log che viene prodotto durante la compilazione del documento. Il fine log  utilizza un sistema di parentesi tonde  che aiutano a capire l'ordine caricamento dei files. Il  frammento di log~\vref{fig:esmpiolog1} segue dalla compilazione del file Mkeyval.tex~. Dopo um preambolo, inizia la compilazione\footnote{Parentesi tonda rossa.} poi viene caricato \textit{babel}~\footnote{O meglio la gestione dei linguaggi supportati}, poi la \textit{classe minimal}\footnote{Parentesi verde} e infine \textit{keyval}. Dopo aver gestito i font la compilazione finisce.
\selectlanguage{english}
\begin{figure}
	\begin{alltt}
		This is pdfTeX, Version 3.14159265-2.6-1.40.15 (TeX Live 2014/W32TeX)
		 (preloaded format=pdflatex 2014.12.27)  5 JAN 2015 22:27
		entering extended mode
		\string\write18 enabled.
		\%&-line parsing enabled.
		**C:/tex/geneatex/test/Mkeyval.tex
		\textcolor{red}{(}c:/tex/geneatex/test/Mkeyval.tex
		LaTeX2e <2014/05/01>
		Babel <3.9l> and hyphenation patterns for 79 languages loaded.
		\textcolor{green}{(}c:/texlive/2014/texmf-dist/tex/latex/base/minimal.cls
		Document Class: minimal 2001/05/25 Standard LaTeX minimal class
		\textcolor{green}{)} \textcolor{blue}{(}c:/texlive/2014/texmf-dist/tex/latex/graphics/keyval.sty
		Package: keyval 2014/10/28 v1.15 key=value parser (DPC)
		\textbackslash{}KV@toks@=\textbackslash{}toks14
		\textcolor{blue}{)}
		No file Mkeyval.aux.
		\string\openout1 = `Mkeyval.aux'.
		
		LaTeX Font Info:    Checking defaults for OML/cmm/m/it on input line 5.
		LaTeX Font Info:    ... okay on input line 5.
		LaTeX Font Info:    Checking defaults for T1/cmr/m/n on input line 5.
		LaTeX Font Info:    ... okay on input line 5.
		LaTeX Font Info:    Checking defaults for OT1/cmr/m/n on input line 5.
		LaTeX Font Info:    ... okay on input line 5.
		LaTeX Font Info:    Checking defaults for OMS/cmsy/m/n on input line 5.
		LaTeX Font Info:    ... okay on input line 5.
		LaTeX Font Info:    Checking defaults for OMX/cmex/m/n on input line 5.
		LaTeX Font Info:    ... okay on input line 5.
		LaTeX Font Info:    Checking defaults for U/cmr/m/n on input line 5.
		LaTeX Font Info:    ... okay on input line 5.
		[1{c:/texlive/2014/texmf-var/fonts/map/pdftex/updmap/pdftex.map}]
		
		(./Mkeyval.aux)
		
		\textcolor{red}{)}
	\end{alltt}
	\captionsetup{name=Esempio}
	\caption{file log}
	\label{fig:esmpiolog1}
\end{figure}
\selectlanguage{italian}
\lstinputlisting[caption={Esempio Keyval 2},%
captionpos=b,label=lst:keyvalesempio2,float]{test/Mkeyval2.tex} 

Altre informazioni si ottengono utilizzando  comando \textit{listfiles} nel preambolo del file~\vref{lst:keyvalesempio2}. Questa opzione  aggiunge al log una sezione  chiamata \textit{File List}~\vref{fig:esmpiolog2} che elenca pacchetti caricati.
\selectlanguage{english}
\begin{figure}
	\begin{alltt}
*File List*
minimal.cls    2001/05/25 Standard LaTeX minimal class
keyval.sty    2014/10/28 v1.15 key=value parser (DPC)
***********	\end{alltt}
	\captionsetup{name=Esempio}
	\caption{file log 2}
	\label{fig:esmpiolog2}
\end{figure}
\lstinputlisting[caption={Esempio Keyval 3},%
captionpos=b,label=lst:keyvalesempio3,float]{test/Mkeyval3.tex} 
\selectlanguage{english}
\begin{lstlisting}[float,caption={Esempio di file dep},label=fig:esmpiodep1,captionpos=b,language=dep]
\RequireVersions{
*{application}{TeX}     {1990/03/25 v3.x}
*{format} {LaTeX2e}     {2014/05/01 v2.e}
*{package}{snapshot}    {2002/03/05 v1.14}
*{class}  {minimal}     {2001/05/25 v0.0}
*{package}{keyval}      {2014/10/28 v1.15}
}
\end{lstlisting}
\selectlanguage{italian}

Un altro metodo è quello di utilizzare il pacchetto \textit{snapshot}. Questo pacchetto, caricato prima di \textit{documentclass}, come nel file~\vref{lst:keyvalesempio3},  genera un elenco,esterno al file log, come nell'esempio~\vref{fig:esmpiodep1} alternativo a quello di \textit{listfiles} che mostra i pacchetti necessari alla compilazione del documento. 
\begin{lstlisting}[float,caption={Esempio di file fls},label=fig:esmpiofls1,captionpos=b,language=fls]
PWD C:/tex/geneatex/test
INPUT c:/texlive/2014/texmf.cnf
INPUT c:/texlive/2014/texmf-dist/web2c/texmf.cnf
INPUT c:/texlive/2014/texmf-var/web2c/pdftex/pdflatex.fmt
INPUT Mkeyval.tex
OUTPUT Mkeyval.log
INPUT c:/texlive/2014/texmf-dist/tex/latex/base/minimal.cls
INPUT c:/texlive/2014/texmf-dist/tex/latex/base/minimal.cls
INPUT c:/texlive/2014/texmf-dist/tex/latex/graphics/keyval.sty
INPUT c:/texlive/2014/texmf-dist/tex/latex/graphics/keyval.sty
INPUT Mkeyval.aux
INPUT Mkeyval.aux
OUTPUT Mkeyval.aux
OUTPUT Mkeyval.pdf
INPUT c:/texlive/2014/texmf-var/fonts/map/pdftex/updmap/pdftex.map
INPUT Mkeyval.aux
INPUT c:/texlive/2014/texmf-dist/fonts/type1/public/amsfonts/cm/cmr10.pfb
\end{lstlisting}

Tramite l'opzione \textit{-recorder}  di \textit{pdflatex} 
viene prodotto un file, di estensione \textit{fls} come quello che è nell'esempio~\vref{fig:esmpiofls1}. Il file ci elenca i processi di input output che avvengono durante la compilazione. Dopo un preambolo inizia la compilazione.
\begin{lstlisting}[float,caption={Esempio Minimal.cls},label=minimacls1,firstline=30,lastline=45,captionpos=b]
% \iffalse meta-comment
%
% Copyright 1993 1994 1995 1996 1997 1998 1999 2000 2001 2002 2003 2004 2005 2006 2007 2008 2009
% The LaTeX3 Project and any individual authors listed elsewhere
% in this file.
%
% This file is part of the LaTeX base system.
% -------------------------------------------
%
% It may be distributed and/or modified under the
% conditions of the LaTeX Project Public License, either version 1.3c
% of this license or (at your option) any later version.
% The latest version of this license is in
%    http://www.latex-project.org/lppl.txt
% and version 1.3c or later is part of all distributions of LaTeX
% version 2005/12/01 or later.
%
% This file has the LPPL maintenance status "maintained".
%
% The list of all files belonging to the LaTeX base distribution is
% given in the file `manifest.txt'. See also `legal.txt' for additional
% information.
%
% The list of derived (unpacked) files belonging to the distribution
% and covered by LPPL is defined by the unpacking scripts (with
% extension .ins) which are part of the distribution.
%
% \fi
%%
%% Minimal LaTeX class file.
%%

\NeedsTeXFormat{LaTeX2e}
\ProvidesClass{minimal}[2001/05/25 Standard LaTeX minimal class]

\renewcommand\normalsize{\fontsize{10pt}{12pt}\selectfont}

\setlength{\textwidth}{6.5in}
\setlength{\textheight}{8in}

\pagenumbering{arabic}  % but no page numbers are printed because:
\pagestyle{empty}       % this is actually already in the kernel

% This documentclass is intended primarily for testing and reference
% purposes; loading it with \LoadClass{minimal} to use it as a base
% class for some other document class is probably a mistake. If you wish
% to start a new document class based on the minimal class, it is better
% to start by copying the *contents* of minimal.cls directly into your
% new class and making suitable modifications. You may, at that point
% also want to start documenting the code using the conventions of the
% doc package, rather than using simple ascii comments as used here.
\end{lstlisting}

Importante, inoltre, è la documentazione allegata al  pacchetto e le informazioni che il 
\foreignlanguage{english}{reverse engineering} fornisce dalla lettura del codice della file. 
\section{Le convenzioni}
 Ora serve un po di vocabolario. 
 \begin{enumerate}
 	\item \textbf{Dipendenza} di x da y: il file y viene caricato dal file x;
 	\item \textbf{Compatibilità} di x e y: il file x e il file y non hanno nessun legame reciproco e possono venire caricati in un ordine qualsiasi;
 	\item \textbf{Indipendenza} di x da y: il file y, anche se invocato, non viene caricato da y
 	\item \textbf{Incompatibilità} di x e y; se x e y vengono invocati entrambi la compilazione non va a buon fine
 	\item \textbf{Sequenzialità } di x e y: i pacchetti x e y vanno caricati in una sequenza precisa
 	\item \textbf{Esclusione} fra x e y: se viene caricato x non viene caricato y. Ma non è detto che valga il viceversa.
 \end{enumerate}
\begin{figure}
	\centering
	\begin{tikzpicture}

	\matrix(tabella)[matrix of nodes,column sep=1em, row sep=.4em]
	{
		\node[testo,align=center] (a){Tipo};&&\node[testo,align=center] (b){Simbolo};\\
		\node[testo] (1){Principale}; &&\node[main node] (2){test}; \\
		\node[testo] (3) {Secondario};&&\node[main2 node] (4){test};\\
		\node[testo] (5) {Complesso}; &&\node[complesso node] (6){test};\\
		\node[testo] (7) {Interno}; &&\node[internal node] (8){test};\\
		\node[testo] (9) {Driver};&&\node[driver node] (10){test};\\
		\node[testo] (11) {Cfg};&&\node[cfg node] (12){test};\\
		\node[testo] (13) {Formato};&&\node[formato node] (14){test};\\
	};
	\end{tikzpicture}
	\captionsetup{name=Legenda}
	\caption{Simboli grafici}
	\label{fig:didascaliagrafici}
\end{figure}
Le dipendenze sono rappresentate anche graficamente come un insieme di nodi  
collegati tramite frecce. Il nodo principale  carica in sequenza altri pacchetti. La legenda~\vref{fig:didascaliagrafici} elenca i vari tipi. Abbiamo un nodo principale che carica  in sequenza i secondari. Un nodo è interno quando il pacchetto è chiamato solo per il funzionamento di un altro per esempio "ifetex". Un nodo è complesso quando la sua descrizione è già avvenuta 'in un altra parte e occuperebbe molto spazio nel grafico. Una caso a parte sono i dover. Un driver è un pacchetto che cambia in base al contesto in cui avviene la compilazione, per esempio pdftex.def al posto di xetex.def  etc.  Un pacchetto è di configurazione se fornisce le impostazioni di base del pacchetto.  