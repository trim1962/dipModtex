\documentclass{standalone}
\usepackage[autosize]{dot2texi}
\usepackage{tikz}
\usetikzlibrary{shapes,arrows,positioning}
%\usepackage[active,tightpage]{preview}
%\setlength\PreviewBorder{0pt}%
%\PreviewEnvironment{dot2tex}
\begin{document}
 \begin{tikzpicture}%[scale=0.8]
 
   \tikzset{base/.style={align=center,text=black}, }	
   %\tikzset{testobase/.style={minimum  width=1.0em, }, }
   \tikzset{testo/.style={,align=left}, }
   \tikzset{main node/.style={base, draw=red, rounded corners}, }
   \tikzset{main2 node/.style={base, draw=green,rounded corners}, }
   \tikzset{complesso node/.style={base, draw=blue}, }
   \tikzset{internal node/.style={base, draw=magenta,dashed,rounded corners}, }
   \tikzset{driver node/.style={base,dashed,draw=blue!20}, }
   \tikzset{formato node/.style={base,dotted}, }
   \tikzset{cfg node/.style={base,trapezium,,trapezium left angle=70,trapezium right angle=-70,draw=brown}, }
   \tikzset{linea/.style={-triangle 90},}
 \begin{dot2tex}[format=tikz,scale=1,styleonly]
 digraph G{
 	 size = "8.3,11.7";
 	center="true";
  	rankdir="LR";

64002 [style="main node",texlbl="pgfmathfunctions.code.tex"]
640020 [style="main2  node",texlbl="pgfmathfunctions.basic.code.tex"]
640021 [style="main2  node",texlbl="pgfmathfunctions.trigonometric.code.tex"]
640022 [style="main2  node",texlbl="pgfmathfunctions.random.code.tex"]
640022 [style="main2  node",texlbl="pgfmathfunctions.comparison.code.tex "]
640023 [style="main2  node",texlbl="pgfmathfunctions.base.code.tex "]
640024 [style="main2  node",texlbl="pgfmathfunctions.round.code.tex "]
640025 [style="main2  node",texlbl="pgfmathfunctions.misc.code.tex "]
640026 [style="main2  node",texlbl="pgfmathfunctions.integerarithmetics.code.tex"]


64002->640020
64002->640021
64002->640022
64002->640023
64002->640024
64002->640025
64002->640026

 }
 \end{dot2tex}
 %% 
 \end{tikzpicture}
\end{document}