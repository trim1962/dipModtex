\documentclass[10pt,a4paper,oneside]{book}
\usepackage{geometry}
\geometry{a4paper,top=1.5cm,bottom=1.5cm,left=2cm,right=2cm}
\usepackage{dot2texi}
\usepackage[T1]{fontenc}
\usepackage[utf8]{inputenc}
\usepackage[osf,sc]{mathpazo}
\usepackage[scaled=0.8]{beramono}
\renewcommand{\sfdefault}{iwona}
\usepackage[final]{microtype}
\usepackage[english,italian]{babel}
\usepackage{imakeidx}
\usepackage{fix2col}
\indexsetup{level=\chapter,toclevel=chapter,noclearpage}
\makeindex[options=-s oldclaudio.sti]
\makeindex[name=emulato,title=Emulati,options=-s oldclaudio.sti]
\makeindex[name=incomp,title=Incompatibili]
\makeindex[name=Warning,title=Warning,options=-s oldclaudio.sti]
\makeindex[name=usato,title=Usato,options=-s oldclaudio.sti]
\makeindex[name=dipende,title=Dipende,options=-s oldclaudio.sti]
\makeindex[name=conffile,title=File di configurazione,options=-s oldclaudio.sti]
\makeindex[name=drvfile,title=Driver,options=-s oldclaudio.sti]
\usepackage[italian]{varioref}
\usepackage{enumitem}
\setlist{nolistsep}
\newcommand{\Mindex}[2]{\index[emulato]{#1!#2}}
\newcommand{\Iindex}[2]{\index[incomp]{#1!#2}}
\newcommand{\Windex}[2]{\index[Warning]{#1!#2}}
\newcommand{\Uindex}[2]{\index[usato]{#1!#2}\index[dipende]{#2!#1}}
\newcommand{\Cindex}[2]{\index[conffile]{#2!#1}}
\newcommand{\Dindex}[2]{\index[drvfile]{#1!#2}}
\usepackage{tikz}
\tikzset{base/.style={align=center,text=black}, }	
%\tikzset{testobase/.style={minimum  width=1.0em, }, }
\tikzset{testo/.style={align=left,text width=15.5mm]}, }
\tikzset{main node/.style={base, draw=red, rounded corners}, }
\tikzset{main2 node/.style={base, draw=green,rounded corners}, }
\tikzset{complesso node/.style={base, draw=blue}, }
\tikzset{internal node/.style={base, draw=magenta,dashed,rounded corners}, }
\tikzset{driver node/.style={base,dashed,draw=blue!20}, }
\tikzset{formato node/.style={base,dotted}, }
\tikzset{cfg node/.style={base,trapezium,,trapezium left angle=70,trapezium right angle=-70,draw=brown}, }
\tikzset{linea/.style={-triangle 90},}

\usetikzlibrary{external}
\tikzexternalize[prefix=figures/]

\usetikzlibrary{shapes,arrows,positioning,shapes.misc,shapes.geometric}
\usetikzlibrary{matrix}
\usepackage{caption}
\captionsetup[figure]{name=Grafico}
\usepackage{listings}
\addto\captionsitalian{
	\renewcommand{\lstlistingname}{Listato}}
\addto\captionsitalian{%
	\renewcommand{\lstlistlistingname}{Elenco dei listati}}
\lstdefinelanguage{dep}
{keywords={application,format,package,class,file,RequireVersions},%
	sensitive=true,%
	alsoletter={\$},%
	comment=[l]{\#},%
	string=[b]",%
	string=[b]'%
}
\lstdefinelanguage{fls}
{keywords={PWD,INPUT,OUTPUT},%
	sensitive=true,%
	alsoletter={\$},%
	comment=[l]{\#},%
	string=[b]",%
	string=[b]'%
}
\author{Claudio Duchi}
\title{Dipendenze \LaTeX}
\usepackage{alltt}
\usepackage{xcolor}
\listfiles
\begin{document}

%	\begin{figure}
%		\centering
%		\begin{tikzpicture}%[scale=0.8]
%		
%		\begin{dot2tex}[options= --autosize --cache --styleonly]
%		digraph G {
%			rankdir="LR";
%			ranksep="1.0 equally"
%			0 [style="complesso node",texlbl="book"]
%			1 [style="main node",texlbl="suftesi"]
%			4 [style="main2 node",texlbl="enumitem"]
%			7 [style="main2 node",texlbl="multicol"]
%			8 [style="main2 node",texlbl="emptypage"]
%			9 [style="main2 node",texlbl="textcase"]
%			10 [style="internal node",texlbl="ifxetex"]
%			12 [style="complesso node",texlbl="cclicenses"]
%			14 [style="main2 node",texlbl="sostitutefont"]
%			17   [style="main2 node",texlbl="extramarks"]
%			18   [style="main2 node",texlbl="fancyhdr"]
%			19  [style="main2 node",texlbl="titletoc"]
%			20  [style="main2 node",texlbl="fixltxhyph"]
%			21  [style="complesso node",texlbl="mathpazo"]
%			1->{0 7 8 9 10 12 14 14 17 18 19 20 21}
%			1->4[labelfloat= true, label = "[inline]"]
%		}
%		\end{dot2tex}
%		%% 
%		\end{tikzpicture}
%		\caption{Suftesi 1}
%		\label{fig:Suftesi1}
%		\end{figure}\clearpage
%	\begin{figure}
%		\centering
%		\begin{tikzpicture}%[scale=0.8]
%		\begin{dot2tex}[options= --autosize --cache --styleonly]
%		digraph G
%		{
%			ranksep="1.0 equally"
%			
%			1 [style="main node",texlbl="suftesi"]
%			2 [style="complesso node",texlbl="geometry"]
%			201 [style="internal node",texlbl="keyvall"]
%			3 [style="complesso node",texlbl="xkeyval"] 
%			5 [style="complesso node",texlbl="caption"]
%			6 [style="complesso node",texlbl="color"]
%			10 [style="internal node",texlbl="ifxetex"]
%			11 [style="complesso node",texlbl="microtype"]
%			13 [style="complesso node",texlbl="fontenc"]
%			15 [style="complesso node",texlbl="crop"]
%			151 [style="complesso node",texlbl="graphics"]   
%			16   [style="main2 node",texlbl="titlesec"]
%			1601[style="cfg node",texlbl="ttlkey.cfg"]
%			22  [style="main2 node",texlbl="beramono"]
%			1->{2 3 5 6  10 16 22 }
%			1->11[labelfloat= true, label = "[final]"]
%			1->13[labelfloat= true, label = "[LGR,T1]"]
%			1->15[labelfloat= true, label = "[A4,CAM,CENTER]"]
%			2->{10 201 }
%	        3->201[labelfloat= true, label = "blocca"]
%			5->201
%			11->201
%			15->{6 151}
%			
%			16->1601
%			1601->201
%			22->201
%		}
%		\end{dot2tex}
%		%% 
%		\end{tikzpicture}
%		\caption{Suftesi 2}
%		\label{fig:Suftesi2}
%	\end{figure}\clearpage
	\begin{figure}
		\centering
		\begin{tikzpicture}%[scale=0.8]
		\begin{dot2tex}[options= --autosize --cache --styleonly]
		digraph G
		{
			ranksep="0.2 equally"
			
		ratio=compress;
			rankdir="LR";
	
			10 [style="main node",texlbl="beamerbaserequires"]
			100 [style="main2 node",texlbl="beamerbasecompatibility"]
			101 [style="main2 node",texlbl="beamerbasefont"]
			1010 [style="internal node",texlbl="amssymb"]
			1011 [style="internal node",texlbl="sansmathaccent"]
			10110 [style="internal node",texlbl="filehook"]
			
			102 [style="main2 node",texlbl="beamerbasetranslator"]
			1020 [style="complesso node",texlbl="translator"]
			103 [style="main2 node",texlbl="beamerbasemisc"]
			104 [style="main2 node",texlbl="beamerbasetwoscreens"]
			105 [style="main2 node",texlbl="beamerbaseoverlay"]
			106 [style="main2 node",texlbl="beamerbasetitle"]
			107 [style="main2 node",texlbl="beamerbasesection"]
			108 [style="main2 node",texlbl="beamerbaseframe"]
			109 [style="main2 node",texlbl="beamerbaseverbatim"]
			110 [style="main2 node",texlbl="beamerbaseframesize"]
			111 [style="main2 node",texlbl="beamerbaseframecomponents"]
			112 [style="main2 node",texlbl="beamerbasecolor"]
			113 [style="main2 node",texlbl="beamerbasetoc"]
			114 [style="main2 node",texlbl="beamerbasetemplates"]
			1140 [style="internal node",texlbl="beamerbaseauxtemplates"]
			11400 [style="internal node",texlbl="beamerbaseboxes"]
			115 [style="main2 node",texlbl="beamerbaselocalstructure"]
			1150 [style="internal node",texlbl="enumerate"]
			116 [style="main2 node",texlbl="beamerbasenavigation"]
			117 [style="main2 node",texlbl="beamerbasetheorems"]
			1170 [style="complesso node",texlbl="amsmath"]
			1171 [style="complesso node",texlbl="amsthm"]
			118 [style="complesso node",texlbl="beamerbasethemes"]
			119 [style="main2 node",texlbl="beamerbasenotes"] 	
			101->{1010 1011}
			1011->10110
			102->1020
			10->{100 101 102 103 104 105 106 107 108 109 110 111 112 113 114 115 116 117 118 119}
			114->1140
			1140->11400
			115->1150
			117->{1170 1171}
			}
		\end{dot2tex}
		%% 
		\end{tikzpicture}
		\caption{Beamer beamerbaserequires}
		\label{fig:beamer2}
	\end{figure}\clearpage
	\begin{figure}
		\centering
		\begin{tikzpicture}%[scale=0.8]
		\begin{dot2tex}[options= --autosize --cache --styleonly]
		digraph G
		{
		//	size = "8.3,11.7";
		//	center="true";
			ranksep="0.2 equally"
			rankdir="LR";
			64 [style="main node",texlbl="pgfcore.code.tex"]
			640 [style="complesso node",texlbl="pgfmath.code.tex"]
			641 [style="main2 node",texlbl="pgfcorepoints.code.tex"]
			642 [style="main2 node",texlbl="pgfcorepathconstruct.code.tex"]
			643 [style="main2 node",texlbl="pgfcorepathusage.code.tex"]
			644 [style="main2 node",texlbl="pgfcorescopes.code.tex"]
			645 [style="main2 node",texlbl="pgfcoregraphicstate.code.tex"]
			646[style="main2 node",texlbl="pgfcoretransformations.code.tex"]
			647 [style="main2 node",texlbl="pgfcorequick.code.tex"]
			648 [style="main2 node",texlbl="pgfcoreobjects.code.tex"]
			649 [style="main2 node",texlbl="pgfcorepathprocessing.code.tex"]
			6410 [style="main2 node",texlbl="pgfcorearrows.code.tex"]
			6411 [style="main2 node",texlbl="pgfcoreshade.code.tex"]
			6412 [style="main2 node",texlbl="pgfcoreimage.code.tex"]
			64120 [style="internal node",texlbl="pgfcoreexternal.code.tex"]
			6413 [style="main2 node",texlbl="pgfcorelayers.code.tex"]
			6414 [style="main2 node",texlbl="pgfcoretransparency.code.tex"]
			6415 [style="main2 node",texlbl="pgfcorepatterns.code.tex"]
			64->{640 641 642 643 644 645 646 647 648 649 6410 6411 6412 6413 6414 6415}
		 6412->64120
		}
		\end{dot2tex}
		%% 
		\end{tikzpicture}
		\caption{Beamer pgfcore.code.tex}
		\label{fig:beamer3}
	\end{figure}\clearpage
	\begin{figure}
		\centering
		\begin{tikzpicture}%[scale=0.8]
		\begin{dot2tex}[options= --autosize --cache --styleonly]
		digraph G
		{
		ranksep="1.0 equally"
			root=6400;
			640 [style="main node",texlbl="pgfmath.code.tex"]
			6400 [style="main2 node",texlbl="pgfmathcalc.code.tex"]
			64000 [style="internal node",texlbl="pgfmathutil.code.tex"]
			64001 [style="internal node",texlbl="pgfmathparser.code.tex"]
			64002 [style="complesso node",texlbl="pgfmathfunctions.code.tex"]
			6401 [style="main2 node",texlbl="pgfmathfloat.code.tex"]
			640->{6400 6401}
			6400->{64000 64001 64002}
	}
		\end{dot2tex}
		
		%% 
		\end{tikzpicture}
		\caption{Beamer pgfmath.code.tex}
		\label{fig:beamer4}
	\end{figure}\clearpage
	\begin{figure}
		\centering
		\begin{tikzpicture}%[scale=0.8]
		\begin{dot2tex}[options= --autosize --cache --styleonly]
		digraph G
		{
	
	ranksep="1.0 equally"
		
			61 [style="main node",texlbl="pgfsys"]
			610 [style="main2 node",texlbl="pgfrcs"]
			6100 [style="internal node",texlbl="pgfutil-common.tex"]
			6101 [style="cfg node",texlbl="pgfutil-latex.def"]
			61010 [style="internal node",texlbl="everyshi"]
			6102 [style="internal node",texlbl="pgfrcs.code.tex"]
			611 [style="main2 node",texlbl="pgfsyssoftpath.code.tex"]
			612 [style="main2 node",texlbl="pgfsyaprotocol.code.tex"]
			613 [style="main2 node",texlbl="pgfsys.code.tex"]
			6130  [style="internal node",texlbl="pgfkeys.code.tex"] 
			61300 [style="internal node",texlbl="ppgfkeysfiltered.code.tex"]  	
			6131 [style="cfg node",texlbl="pgf.cfg"] 
			root="61";
			61->{610 611 612 613 }
		    610->{6100 6101 6102}
		    613->{6130 6131}
			6101->61010
			6130->61300
		}
		\end{dot2tex}
		\end{tikzpicture}
		\caption{Beamer pgfsys}
		\label{fig:beamer5}
	\end{figure}\clearpage
	\begin{figure}
		\centering
		\begin{tikzpicture}%[scale=0.8]
		\begin{dot2tex}[options= --autosize --cache --styleonly]
		digraph G
		{
			size = "8.3,11.7";
			ranksep="0.2 equally"
			center="true";
			rankdir="LR";
			1020 [style="main node",texlbl="translator"]
			10200 [style="cfg node",texlbl="translator-basic-dictionary"]
			10201 [style="cfg node",texlbl="translator-bibliography-dictionary"]
			10202 [style="cfg node",texlbl="translator-environment-dictionary"]
			10203 [style="cfg node",texlbl="translator-months-dictionary"]
			10204 [style="cfg node",texlbl="translator-numbers-dictionary"]
			10205 [style="cfg node",texlbl="translator-theorem-dictionary"]
			10206 [style="internal node",texlbl="keyval"]
			10207 [style="cfg node",texlbl="translator-language-mappings"]
			1020->{10200 10201 10202 10203 10204 10205 10206 10207}		
		}
		\end{dot2tex}
		\end{tikzpicture}
		\caption{Beamer translator}
		\label{fig:beamer6}
	\end{figure}\clearpage
	\begin{figure}
		\centering
		\begin{tikzpicture}%[scale=0.8]
		\begin{dot2tex}[options= --autosize --cache --styleonly]
		digraph G
		{
			size = "8.3,11.7";
			center="true";
			rankdir="LR";
			ranksep="0.2 equally"
			64002 [style="main node",texlbl="pgfmathfunctions.code.tex"]
			640020 [style="main2  node",texlbl="pgfmathfunctions.basic.code.tex"]
			640021 [style="main2  node",texlbl="pgfmathfunctions.trigonometric.code.tex"]
			640022 [style="main2  node",texlbl="pgfmathfunctions.random.code.tex"]
			640022 [style="main2  node",texlbl="pgfmathfunctions.comparison.code.tex "]
			640023 [style="main2  node",texlbl="pgfmathfunctions.base.code.tex "]
			640024 [style="main2  node",texlbl="pgfmathfunctions.round.code.tex "]
			640025 [style="main2  node",texlbl="pgfmathfunctions.misc.code.tex "]
			640026 [style="main2  node",texlbl="pgfmathfunctions.integerarithmetics.code.tex"]
			64002->{640020 640021 640022 640023 640024 640025 640026}
		}
		\end{dot2tex}
		\end{tikzpicture}
		\caption{Beamer pgfmathfunctions.code.tex}
		\label{fig:beamer7}
	\end{figure}\clearpage
	\begin{figure}
		\centering
		\begin{tikzpicture}%[scale=0.8]
		\begin{dot2tex}[options= --autosize --cache --styleonly]
		digraph G
		{
		ranksep="1.0 equally"
			
			118 [style="main node",texlbl="beamerbasethemes"]
			1180 [style="internal node",texlbl="beamerthemedefault"]
			11800 [style="internal node",texlbl="beamerfontthemedefault"]
			11801 [style="internal node",texlbl="beamercolorthemedefault"]
			11802 [style="internal node",texlbl="beamerinnerthemedefault"]
			11803 [style="internal node",texlbl="beamerouterthemedefault"]
			118->1180
			1180->{11800 11801 11802 11803}
		}
		\end{dot2tex}
		\end{tikzpicture}
		\caption{Beamer beamerbasethemes}
		\label{fig:beamer8}
	\end{figure}\clearpage
	\begin{figure}
		\centering
		\begin{tikzpicture}%[scale=0.8]
		\begin{dot2tex}[twopi,format=tikz,scale=1,styleonly,options= --autosize --cache]
		digraph G
		{
			size = "8.3,11.8";
			rankdir="LR";
			ranksep="1.0 equally"
			1 [style="main node",texlbl="beamer.cls"]
			2 [style="main2 node",texlbl="beamerbasercs"] 
			3 [style="main2 node",texlbl="beamerbasemodes"]
			30[style="internal node",texlbl="beamerbasedecode"] 
			
			4 [style="internal node",texlbl=" ifpdf"]
			5 [style="main2 node",texlbl="beamerbaseoptions"]
			50[style="internal node",texlbl="keyvall"]
			6 [style="main2 node",texlbl="pgfcore"]
			60 [style="complesso node",texlbl="graphicx"]
			61 [style="complesso node",texlbl="pgfsys"]
			63 [style="complesso node",texlbl="xcolor"]
			64 [style="complesso node",texlbl="pgfcore.code.tex"]
			
			7 [style="main2 node",texlbl="xxcolor"]
			8 [style="complesso node",texlbl="atbegshi"]
			9 [style="complesso node",texlbl="hyperref"]
			10 [style="complesso node",texlbl="beamerbaserequires"]
			11 [style="complesso node",texlbl="geometry"]
			center="1";
			1->{2 3 4 5 6 7 8 9 10 11}
			6->{50 60 61 63 64 }
			3->30
			5->50
			7->63
		    10->50
			8->4
			60->4
		}
		\end{dot2tex}
		%% 
		\end{tikzpicture}
		\caption{Beamer generale}
		\label{fig:beamer0}
	\end{figure}\clearpage
\begin{figure}
	\centering
	\begin{tikzpicture}%[scale=0.8]
	
	\begin{dot2tex}[options= --autosize --cache --styleonly]
	digraph G
	{
	ranksep="1.0 equally"
		1 [style="main node",texlbl="amsxtra"]
		2 [style="main node",texlbl="amsmath"]
		5 [style="main2 node",texlbl="amsbsy"]
		6 [style="main2 node",texlbl="amsopen"]
		4 [style="main2 node",texlbl="amstext"]
		3 [style="main node",texlbl="amscd"]
		7 [style="main2 node",texlbl="amsgen"]
		1->2
		2->{4 5 6}
		3->7
		4->7
		5->7
		6->7 
	}
	\end{dot2tex}
	\end{tikzpicture}
	\caption{AmsMath}
	\label{fig:AmsMath1}
\end{figure}\clearpage
\begin{figure}
	\centering
	\begin{tikzpicture}
	
	\begin{dot2tex}[options= --autosize --cache --styleonly]
	digraph G
	{
	ranksep="1.0 equally"
		1 [style="main node",texlbl="atbegshi"]
		2 [style="internal node",texlbl="infwarerr"]
		3 [style="internal node",texlbl="ltxcmds"]
		4 [style="internal node",texlbl="ifpdf"] 
		1->{2 3 4}
	}
	\end{dot2tex}
	% 
	\end{tikzpicture}
	\caption{Atbegshi}
	\label{fig:atbegshi1}
\end{figure}\clearpage
\begin{figure}
	\centering
	\begin{tikzpicture}%[scale=0.8]
	\begin{dot2tex}[styleonly,options= --autosize --cache]
	digraph G
	{
	
		ranksep="1.0 equally";
		1 [style="main node",texlbl="bitset"]
		2 [style="internal node",texlbl="infwarerr"]
		3 [style="main2 node",texlbl="intcalc"]
		4 [style="main2 node",texlbl="bigintcalc"]
		5 [style="internal node",texlbl="pdftexcmds"]
		6 [style="internal node",texlbl="ifluatex"]
		7 [style="internal node",texlbl="ltxcmds"]
		8 [style="internal node",texlbl="ifpdf"] 
		1->{2 3 4}
		4->5
		5->{6 7 8 2}
	}
	\end{dot2tex}
	\end{tikzpicture}
	\caption{Bitset}
	\label{fig:bitset1}
\end{figure}\clearpage
\begin{figure}
	\centering
	\begin{tikzpicture}%[scale=0.8]
	
	\begin{dot2tex}[styleonly,options= --autosize --cache]
	digraph G
	{
		ranksep="1.0 equally"
		1 [style="main node",texlbl="cclicenses"];
		2 [style="main2 node",texlbl="rotating"];
		3 [style="internal node",texlbl="ifthen"];
		4 [style="complesso node",texlbl="graphicx"];
		1->2
		2->{3 4}
	}
	\end{dot2tex}
	% 
	\end{tikzpicture}
	\caption{cclicenses}
	\label{fig:cclicenses1}
\end{figure}\clearpage
\begin{figure}
	\centering
	\begin{tikzpicture}%[scale=0.8]
	
	\begin{dot2tex}[styleonly,options= --autosize --cache]
	digraph G
	{
		ranksep="1.0 equally"
		1 [style="main node",texlbl="color"];
		2 [style="cfg node",texlbl="color.cfg"];
		3 [style="driver node",texlbl="xetex.def"];
		1->{2 3}
		2->3
	}
	\end{dot2tex}
	% 
	\end{tikzpicture}
	\caption{Color xetex}
	\label{fig:color2}
\end{figure}\clearpage
\begin{figure}
	\centering
	\begin{tikzpicture}
	\begin{dot2tex}[styleonly,options= --autosize --cache]
	digraph G {
	ranksep="1.0 equally"
		1 [style="main node",texlbl="color"];
		2 [style="cfg node",texlbl="color.cfg"];
		3 [style="driver node",texlbl="pdftex.def"];
		4 [style="internal node",texlbl="infwarerr"];
		5 [style="cfg node",texlbl="supp-pdf.mkii"];
		6 [style="internal node",texlbl="ltcmds"];
		1->{2 3}
		2->3
		3->{4 5}
		4->6
	}
	\end{dot2tex}
	% 
	\end{tikzpicture}
	\caption{Color}
	\label{fig:color1}
\end{figure}\clearpage
\begin{figure}
	\centering
	\begin{tikzpicture}%[scale=0.8]

	\begin{dot2tex}[styleonly,options= --autosize --cache,straightedges]
	digraph G
	{
		ranksep="1.0 equally"
		1 [style="main node",texlbl="crop"];
		2 [style="main2 node",texlbl="color"];
		3 [style="cfg node",texlbl="color.cfg"];
		4 [style="driver node",texlbl="xetex.def"];
		5 [style="main2 node",texlbl="graphics"];
		6 [style="internal node",texlbl="trig"];
		7 [style="cfg node",texlbl="graphics.cfg"];
		8 [style="cfg node",texlbl="crop.cfg"];
		1->{2 5 8}
		2->{3 4}
		5->{4 6 7}
		7->4
		3->4
	}
	\end{dot2tex}
	% 
	\end{tikzpicture}
	\caption{Crop Xetex}
	\label{fig:crop2}
\end{figure}\clearpage
\begin{figure}
	\centering
	\begin{tikzpicture}%[scale=0.8]
		\begin{dot2tex}[styleonly,options= --autosize --cache]
	digraph G
	{
	ranksep="1.0 equally"
		1 [style="main node",texlbl="crop"];
		2 [style="main2 node",texlbl="color"];
		3 [style="cfg node",texlbl="color.cfg"];
		4 [style="driver node",texlbl="pdftex.def"];
		5 [style="internal node",texlbl="infwarerr"];
		6 [style="internal node",texlbl="latexcmds"];
		7 [style="main2 node",texlbl="graphics"];
		8 [style="internal node",texlbl="trig"];
		9 [style="cfg node",texlbl="graphics.cfg"];
		11 [style="cfg node",texlbl="supp-pdf.mkii"];
		15 [style="complesso node",texlbl="epstopdf-base"];
		17 [style="internal node",texlbl="keydefinekys"];
		1->{2 7}
		2->{3 4}
		3->4
		4->{5 11 17}
		5->6
		7->{4 8 9}
		9->{4 15 }
		15->{5 6 17 }
		17->6
	}
	\end{dot2tex}
	%% 
	\end{tikzpicture}
	\caption{Crop}
	\label{fig:crop1}
\end{figure}\clearpage
\begin{figure}
	\begin{tikzpicture}%[scale=0.8]
	\begin{dot2tex}[scale=0.8,options= --autosize --cache --styleonly]
	digraph G
	{
	ranksep="1.0 equally"
		1 [style="main node",texlbl="epstopdf-base"]
		2 [style="cfg node",texlbl="epstopdf-sys.cfg"]
		3 [style="internal node",texlbl="infwarerr"]
		4 [style="internal node",texlbl="grfext"]
		5 [style="internal node",texlbl="pdftexcmds"]
		6 [style="internal node",texlbl="kvoptions"]
		7 [style="internal node",texlbl="keyval"]
		8 [style="internal node",texlbl="kvsetkeys"]
		9 [style="internal node",texlbl="kvoptions-patch"]
		
		10 [style="internal node",texlbl="kvdefinekeys"]
		11 [style="internal node",texlbl="ltxcmds"]
		12 [style="internal node",texlbl="ifpdf"]
		13 [style="internal node",texlbl="ifluatex"]
		14 [style="cfg node",texlbl="epstopdf.cfg"]
		15 [style="internal node",texlbl="etexcmds"]
		
		1->{2 3 4 5 6 14}
		6->{7 8 9 11}
		5->{3 11 12 13 }
		4->{10 3}
		9->15
		10->11
	}
	\end{dot2tex}
	%% 
	\end{tikzpicture}
	\caption{Epstopdf-base}
	\label{fig:epstopdf-base1}
\end{figure}\clearpage
\begin{figure}
	\centering
	\begin{tikzpicture}%[scale=0.8]
	
	\begin{dot2tex}[options= --autosize --cache --styleonly]
	digraph G {
ranksep="1.0 equally"
		1 [style="main node",texlbl="etoolbox.sty"]
		2 [style="internal node",texlbl="etex"]
		3 [style="cfg node",texlbl="etoolbox.def"]
		1->{2 3}
	}
	\end{dot2tex}
	%% 
	\end{tikzpicture}
	\caption{Etoolbox}
	\label{fig:Etoolbox1}
\end{figure}\clearpage
\begin{figure}
	\centering
	\begin{tikzpicture}%[scale=0.8]
	
	\begin{dot2tex}[options= --autosize --cache --styleonly]
	digraph G {
	ranksep="1.0 equally"
		1 [style="main node",texlbl="geometry"]
		
		2 [style="internal node",texlbl="ifpdf"]
		3 [style="internal node",texlbl="keyvall"]
		
		4 [style="internal node",texlbl="ifvtex"]
		5 [style="internal node",texlbl="ifxetex"]
		1->{2 3 4 5}
	}
	\end{dot2tex}
	%% 
	\end{tikzpicture}
	\caption{Geometry}
	\label{fig:geometry1}
\end{figure}\clearpage

\begin{figure}
	\centering
	\begin{tikzpicture}%[scale=0.8]
	\begin{dot2tex}[options= --autosize --cache --styleonly]
	digraph G {
ranksep="1.0 equally"
		1 [style="complesso node",texlbl="epstopdf-base"]
		3 [style="internal node",texlbl="infwarerr"]
		5 [style="internal node",texlbl="pdftexcmds"]
		11 [style="internal node",texlbl="ltxcmds"]
		12 [style="internal node",texlbl="ifpdf"]
		13 [style="internal node",texlbl="ifluatex"]
		14 [style="main node",texlbl="graphics"]
		15 [style="cfg node",texlbl="graphics.cfg"]
		16 [style="driver node",texlbl="pdftex.def"]
		17 [style="internal node",texlbl="trig"]
		18 [style="cfg node",texlbl="supp-pdf-mkii"]
		1->{3 5}
		
		5->{13 3 11 12}	
		14->{ 15 16 17}	
		15->{1 5 16}	
		16->{ 5 3 18 11}		
	}
	\end{dot2tex}
	\end{tikzpicture}
	\caption{Graphics}
	\label{fig:graphics1}
\end{figure}\clearpage
\begin{figure}
	\centering
	\begin{tikzpicture}%[scale=0.8]
	\begin{dot2tex}[options= --autosize --cache --styleonly]
	digraph G {
	ranksep="1.0 equally"
		14 [style="complesso node",texlbl="graphics"]
		7 [style="internal node",texlbl="keyval"]
		20 [style="main node",texlbl="graphicx"]	
		20->{7 14}
	}
	\end{dot2tex}
	\end{tikzpicture}
	\caption{Graphicx  pdflatex}
	\label{fig:graphicx1}
\end{figure}\clearpage
\begin{figure}
	\centering
	\begin{tikzpicture}%[scale=0.8]
	\begin{dot2tex}[options= --autosize --cache --styleonly]
	digraph G {
			ranksep="1.0 equally";
		1 [style="main node",texlbl="microtype"]
		2 [style="internal node",texlbl="keyval"]
		3 [style="cfg node",texlbl="microtype.cfg"]
		4 [style="driver node",texlbl="microtype-pdftex.def"]
		5 [style="driver node",texlbl="mt-cmr.cfg"]
		1->{2 3 4 5}
	}
	\end{dot2tex}
	\end{tikzpicture}
	\caption{Microtype}
	\label{fig:microtype}
\end{figure}\clearpage  
\begin{figure}
	\centering
	\begin{tikzpicture}%[scale=0.8]
	\begin{dot2tex}[options= --autosize --cache --styleonly]
	
	digraph G {
		ranksep="1.0 equally"
		1 [style="main node",texlbl="pdftexcmds"]
		2 [style="internal node",texlbl="infwarerr"]
		3 [style="internal node",texlbl="ifluatex"]
		4 [style="internal node",texlbl="ltxcmds"]
		5 [style="internal node",texlbl="ifpdf"]
		1->{2 3 4 5}
		
	}
	\end{dot2tex}
	\end{tikzpicture}
	\caption{Pdftexcmds}
	\label{fig:Pdftexcmds}
\end{figure}\clearpage
\begin{figure}
	\centering
	\begin{tikzpicture}%[scale=0.8]
	
	
	\begin{dot2tex}[options= --autosize --cache --styleonly]
	digraph G {
ranksep="1.0 equally"
		1 [style="main node",texlbl="xcolor"];
		2 [style="cfg node",texlbl="color.cfg"];
		3 [style="driver node",texlbl="pdftex.def"];
		4 [style="internal node",texlbl="infwarerr"];
		5 [style="cfg node",texlbl="supp-pdf.mkii"];
		6 [style="internal node",texlbl="ltxcmds"];
		1->{2 3}
		2->3
		3->{4 5 6}
	}
	\end{dot2tex}
	\end{tikzpicture}
	\caption{Xcolor}
	\label{fig:xcolor1}
\end{figure}\clearpage
\begin{figure}
	\centering
	\begin{tikzpicture}
	
	\begin{dot2tex}[options= --autosize --cache --styleonly]
	digraph G {
	ranksep="1.0 equally"
		1 [style="main node",texlbl="xkeyval"];
		2 [style="internal node",texlbl="xkeyval.tex"];
		3 [style="driver node",texlbl="xkvutils.tex"];
		4 [style="internal node",texlbl="keyval.tex"];
		1->{2 3 4}
		
	}
	\end{dot2tex}
	\end{tikzpicture}
	\caption{Xkeyval}
	\label{fig:xkeyval1}
\end{figure}\clearpage
\begin{figure}
	\centering
	\begin{tikzpicture}%[scale=0.8]
	
	\begin{dot2tex}[options= --autosize --cache --styleonly]
	digraph G
{
		ranksep="1.0 equally"
		1 [style="main node",texlbl="babel"];
		2 [style="internal node",texlbl="switch.def"];
		3 [style="driver node",texlbl="italian.ldf"];
		4 [style="internal node",texlbl="etoolboxs"];
		5 [style="cfg node",texlbl="babel.def"];
		6 [style="internal node",texlbl="etex"];
		1->{2 5}
		5->3
		2->3
		3->4
		4->6
		
	}
	\end{dot2tex}
	% 
	\end{tikzpicture}
	\caption{Babel italian}
	\label{fig:babel1}
\end{figure}\clearpage
\begin{figure}
	\centering
	\begin{tikzpicture}%[scale=0.8]
	
	\begin{dot2tex}[options= --autosize --cache --styleonly]
	digraph G
	{
	ranksep="1.0 equally"
		1 [style="main node",texlbl="fourier"]
		2 [style="internal node",texlbl="fourier-orns"]
		3 [style="internal node",texlbl="textcomp"]
		4 [style="internal node",texlbl="[T1]fontenv"]
		5 [style="cfg node",texlbl="ts1enc.def"]
		6 [style="cfg node",texlbl="t1enc.def"]
		1->{2 3 4}
		3->5
		4->6
	}
	\end{dot2tex}
	%% 
	\end{tikzpicture}
	\caption{Fourier}
	\label{fig:fourier1}
\end{figure}\clearpage
\begin{figure}
	\centering
	\begin{tikzpicture}%[scale=0.8]
	\begin{dot2tex}[options= --autosize --cache --styleonly]
	digraph G {
		ranksep="1.0 equally"
		14 [style="main node",texlbl="graphics"]
		15 [style="cfg node",texlbl="graphics.cfg"]
		16 [style="driver node",texlbl="xetex.def"]
		17 [style="internal node",texlbl="trig"]
		14->{15 16 17}
		15->16
		}
	\end{dot2tex}
	\end{tikzpicture}
	\caption{Graphics Xelatex}
	\label{fig:graphics2}
\end{figure}\clearpage
\begin{figure}
	\centering
	\begin{tikzpicture}
	
\begin{dot2tex}[options= --autosize --cache --styleonly]
digraph G
{
	ranksep="1.0 equally"
		7 [style="internal node",texlbl="keyval"]
		14 [style="main2 node",texlbl="graphics"]
		15 [style="cfg node",texlbl="graphics.cfg"]
		16 [style="driver node",texlbl="xetex.def"]
		17 [style="internal node",texlbl="trig"]
		20 [style="main node",texlbl="graphicx"]
		14->{15 16 17}
	    20->{7 14}	
	}
	\end{dot2tex}
	
	\end{tikzpicture}
	\caption{Graphicx Xelatex}
	\label{fig:graphicx2}
\end{figure}\clearpage
\begin{figure}
	\centering
	\begin{tikzpicture}%[scale=0.8]
\begin{dot2tex}[options= --autosize --cache --styleonly]
digraph G {
	ranksep="1.0 equally"
		1 [style="main node",texlbl="xcolor"];
		2 [style="cfg node",texlbl="color.cfg"];
		3 [style="driver node",texlbl="pdftex.def"];
		4 [style="internal node",texlbl="infwarerr"];
		5 [style="cfg node",texlbl="supp-pdf.mkii"];
		6 [style="internal node",texlbl="ltxcmds"];
		1->{2 3}
		2->3
		3->{4 5 6}
	}
	\end{dot2tex}
	\end{tikzpicture}
	\caption{Xcolor}
	\label{fig:xcolor1}
\end{figure}\clearpage
test
\end{document}	