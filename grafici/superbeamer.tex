\documentclass[10pt,a4paper,oneside]{book}
 \usepackage[debug,autosize]{dot2texi}
\usepackage[T1]{fontenc}
\usepackage[utf8]{inputenc}
\usepackage[osf,sc]{mathpazo}
\usepackage[scaled=0.8]{beramono}
\renewcommand{\sfdefault}{iwona}
\usepackage[final]{microtype}
\usepackage[english,italian]{babel}
\usepackage{tikz}
  \tikzset{base/.style={align=center,text=black}, }	
  %\tikzset{testobase/.style={minimum  width=1.0em, }, }
  \tikzset{testo/.style={align=left,text width=15.5mm]}, }
  \tikzset{main node/.style={base, draw=red, rounded corners}, }
  \tikzset{main2 node/.style={base, draw=green,rounded corners}, }
  \tikzset{complesso node/.style={base, draw=blue}, }
  \tikzset{internal node/.style={base, draw=magenta,dashed,rounded corners}, }
  \tikzset{driver node/.style={base,dashed,draw=blue!20}, }
  \tikzset{formato node/.style={base,dotted}, }
  \tikzset{cfg node/.style={base,trapezium,,trapezium left angle=70,trapezium right angle=-70,draw=brown}, }
  \tikzset{linea/.style={-triangle 90},}
   	\usetikzlibrary{external}


\usetikzlibrary{shapes,arrows,positioning,shapes.misc,shapes.geometric}
\usetikzlibrary{matrix}
\usepackage{caption}
\captionsetup[figure]{name=Grafico}
\listfiles
\begin{document}
		\begin{figure}
	\centering
		\begin{tikzpicture}
	
		\matrix(tabella)[matrix of nodes,column sep=1em, row sep=.4em]
		{
			\node[testo,align=center] (a){Tipo};&&\node[testo,align=center] (b){Simbolo};\\
			\node[testo] (1){Principale}; &&\node[main node] (2){test}; \\
			\node[testo] (3) {Secondario};&&\node[main2 node] (4){test};\\
			\node[testo] (5) {Complesso}; &&\node[complesso node] (6){test};\\
			\node[testo] (7) {Interno}; &&\node[internal node] (8){test};\\
			\node[testo] (9) {Driver};&&\node[driver node] (10){test};\\
			\node[testo] (11) {Cfg};&&\node[cfg node] (12){test};\\
			\node[testo] (13) {Formato};&&\node[formato node] (14){test};\\
		};
		\end{tikzpicture}
		\captionsetup{name=Legenda}
		\caption{grafici}
		\label{fig:didascaliagrafici}
	\end{figure}
	\begin{figure}
		\centering
		\begin{tikzpicture}%[scale=0.8]
		
		
		\begin{dot2tex}[format=tikz,scale=1,styleonly]
		digraph G{
			size = "8.3,11.8";
			rankdir="LR";
			1 [style="main node",texlbl="beamer.cls"]
			2 [style="main2 node",texlbl="beamerbasercs"] 
			3 [style="main2 node",texlbl="beamerbasemodes"]
			30[style="internal node",texlbl="beamerbasedecode"] 
			
			4 [style="internal node",texlbl=" ifpdf"]
			5 [style="main2 node",texlbl="beamerbaseoptions"]
			50[style="internal node",texlbl="keyvall"]
			6 [style="main2 node",texlbl="pgfcore"]
			60 [style="complesso node",texlbl="graphicx"]
			61 [style="complesso node",texlbl="pgfsys"]
			63 [style="complesso node",texlbl="xcolor"]
			64 [style="complesso node",texlbl="pgfcore.code.tex"]
			
			7 [style="main2 node",texlbl="xxcolor"]
			8 [style="complesso node",texlbl="atbegshi"]
			9 [style="complesso node",texlbl="hyperref"]
			10 [style="complesso node",texlbl="beamerbaserequires"]
			1->2
			1->3
			3->30
			1->4
			1->5
			5->50
			1->6
			6->60
			6->61
			
			6->50
			6->63
			6->64
			
			7->63
			1->7
			1->8
			1->9
			1->10
			10->50
			8->4
			60->4
		}
		\end{dot2tex}
		%% 
		\end{tikzpicture}
		\caption{Beamer generale}
		\label{fig:beamer0}
	\end{figure}
	\begin{figure}
		\centering
		\begin{tikzpicture}%[scale=0.8]
		\begin{dot2tex}[format=tikz,scale=.8,styleonly]
		digraph G{
			size = "8.3,11.7";
			center="true";
			rankdir="LR";
			10 [style="main node",texlbl="beamerbaserequires"]
			100 [style="main2 node",texlbl="beamerbasecompatibility"]
			101 [style="main2 node",texlbl="beamerbasefont"]
			1010 [style="internal node",texlbl="amssymb"]
			102 [style="main2 node",texlbl="beamerbasetranslator"]
			1020 [style="complesso node",texlbl="translator"]
			103 [style="main2 node",texlbl="beamerbasemisc"]
			104 [style="main2 node",texlbl="beamerbasetwoscreens"]
			105 [style="main2 node",texlbl="beamerbaseoverlay"]
			106 [style="main2 node",texlbl="beamerbasetitle"]
			107 [style="main2 node",texlbl="beamerbasesection"]
			108 [style="main2 node",texlbl="beamerbaseframe"]
			109 [style="main2 node",texlbl="beamerbaseverbatim"]
			110 [style="main2 node",texlbl="beamerbaseframesize"]
			111 [style="main2 node",texlbl="beamerbaseframecomponents"]
			112 [style="main2 node",texlbl="beamerbasecolor"]
			113 [style="main2 node",texlbl="beamerbasetoc"]
			114 [style="main2 node",texlbl="beamerbasetemplates"]
			1140 [style="internal node",texlbl="beamerbaseauxtemplates"]
			11400 [style="internal node",texlbl="beamerbaseboxes"]
			115 [style="main2 node",texlbl="beamerbaselocalstructure"]
			1150 [style="internal node",texlbl="enumerate"]
			116 [style="internal node",texlbl="beamerbasenavigation"]
			117 [style="main2 node",texlbl="beamerbasetheorems"]
			1170 [style="complesso node",texlbl="amsmath"]
			1171 [style="complesso node",texlbl="amsthm"]
			118 [style="main2 node",texlbl="beamerbasethemes"]
			10->100
			10->101
			101->1010
			10->102
			102->1020
			10->103
			10->104
			10->105
			10->106
			10->107
			10->108
			10->109
			10->110
			10->111
			10->112
			10->113
			10->114
			114->1140
			1140->11400
			10->115
			115->1150
			10->116
			10->117
			117->1170
			117->1171
			10->118
			
		}
		\end{dot2tex}
		%% 
		\end{tikzpicture}
		\caption{Beamer beamerbaserequires}
		\label{fig:beamer2}
	\end{figure}
	\begin{figure}
		\centering
		\begin{tikzpicture}%[scale=0.8]
		\begin{dot2tex}[format=tikz,scale=.9,styleonly]
		digraph G{
			size = "8.3,11.7";
			center="true";
			rankdir="LR";
			64 [style="main node",texlbl="pgfcore.code.tex"]
			640 [style="complesso node",texlbl="pgfmath.code.tex"]
			641 [style="main2 node",texlbl="pgfcorepoints.code.tex"]
			642 [style="main2 node",texlbl="pgfcorepathconstruct.code.tex"]
			643 [style="main2 node",texlbl="pgfcorepathusage.code.tex"]
			644 [style="main2 node",texlbl="pgfcorescopes.code.tex"]
			645 [style="main2 node",texlbl="pgfcoregraphicstate.code.tex"]
			646[style="main2 node",texlbl="pgfcoretransformations.code.tex"]
			647 [style="main2 node",texlbl="pgfcorequick.code.tex"]
			648 [style="main2 node",texlbl="pgfcoreobjects.code.tex"]
			649 [style="main2 node",texlbl="pgfcorepathprocessing.code.tex"]
			6410 [style="main2 node",texlbl="pgfcorearrows.code.tex"]
			6411 [style="main2 node",texlbl="pgfcoreshade.code.tex"]
			
			6412 [style="main2 node",texlbl="pgfcoreimage.code.tex"]
			64120 [style="internal node",texlbl="pgfcoreexternal.code.tex"]
			
			6413 [style="main2 node",texlbl="pgfcorelayers.code.tex"]
			6414 [style="main2 node",texlbl="pgfcoretransparency.code.tex"]
			6415 [style="main2 node",texlbl="pgfcorepatterns.code.tex"]
			64->640
			64->641
			64->642
			64->643
			64->644
			64->645
			64->646
			64->647
			64->648
			64->649
			64->6410
			64->6411
			64->6412
			6412->64120
			64->6413
			64->6414
			64->6415
			
		}
		\end{dot2tex}
		%% 
		\end{tikzpicture}
		\caption{Beamer pgfcore.code.tex}
		\label{fig:beamer3}
	\end{figure}
	\begin{figure}
		\centering
		\begin{tikzpicture}%[scale=0.8]
		\begin{dot2tex}[format=tikz,scale=.8,styleonly]
		digraph G{
			size = "8.3,11.7";
			center="true";
			rankdir="LR";
			640 [style="main node",texlbl="pgfmath.code.tex"]
			6400 [style="main2 node",texlbl="pgfmathcalc.code.tex"]
			64000 [style="internal node",texlbl="pgfmathutil.code.tex"]
			64001 [style="internal node",texlbl="pgfmathparser.code.tex"]
			64002 [style="complesso node",texlbl="pgfmathfunctions.code.tex"]
			
			6401 [style="main2 node",texlbl="pgfmathfloat.code.tex"]
			640->6400
			6400->64000
			6400->64001
			6400->64002
			
			640->6401
		}
		\end{dot2tex}
		
		%% 
		\end{tikzpicture}
		\caption{Beamer pgfmath.code.tex}
		\label{fig:beamer4}
	\end{figure}
\begin{figure}
	\centering
	\begin{tikzpicture}%[scale=0.8]
	\begin{dot2tex}[format=tikz,scale=1,styleonly]
	digraph G{
		size = "8.3,11.8";
		rankdir="LR";
		61 [style="main node",texlbl="pgfsys"]
		610 [style="main2 node",texlbl="pgfrcs"]
		6100 [style="internal node",texlbl="pgfutil-common.tex"]
		6101 [style="cfg node",texlbl="pgfutil-latex.def"]
		
		61010 [style="internal node",texlbl="everyshi"]
		6102 [style="internal node",texlbl="pgfrcs.code.tex"]
		611 [style="main2 node",texlbl="pgfsyssoftpath.code.tex"]
		612 [style="main2 node",texlbl="pgfsyaprotocol.code.tex"]
		613 [style="main2 node",texlbl="pgfsys.code.tex"]
		6130  [style="internal node",texlbl="pgfkeys.code.tex"] 
		61300 [style="internal node",texlbl="ppgfkeysfiltered.code.tex"]  	
		6131 [style="cfg node",texlbl="pgf.cfg"] 
		
		61->610
		610->6100
		610->6101
		6101->61010
		610->6102
		61->611
		61->612
		61->613
		613->6130
		6130->61300
		613->6131
	}
	\end{dot2tex}
	\end{tikzpicture}
	\caption{Beamer pgfsys}
	\label{fig:beamer5}
\end{figure}
	\begin{figure}
		\centering
		\begin{tikzpicture}%[scale=0.8]
		\begin{dot2tex}[format=tikz,scale=1,styleonly]
		digraph G{
			size = "8.3,11.7";
			center="true";
			rankdir="LR";
			1020 [style="main node",texlbl="translator"]
			10200 [style="cfg node",texlbl="translator-basic-dictionary"]
			10201 [style="cfg node",texlbl="translator-bibliography-dictionary"]
			10202 [style="cfg node",texlbl="translator-environment-dictionary"]
			10203 [style="cfg node",texlbl="translator-months-dictionary"]
			10204 [style="cfg node",texlbl="translator-numbers-dictionary"]
			10205 [style="cfg node",texlbl="translator-theorem-dictionary"]
			10206 [style="internal node",texlbl="keyval"]
			10207 [style="cfg node",texlbl="translator-language-mappings"]
			1020->10200
			1020->10201
			1020->10202
			1020->10203
			1020->10204
			1020->10205
			1020->10206
			1020->10207
		}
		\end{dot2tex}
		\end{tikzpicture}
		\caption{Beamer translator}
		\label{fig:beamer6}
	\end{figure}
	\begin{figure}
		\centering
		\begin{tikzpicture}%[scale=0.8]
		\begin{dot2tex}[format=tikz,scale=1,styleonly]
		digraph G{
			size = "8.3,11.7";
			center="true";
			rankdir="LR";
			
			64002 [style="main node",texlbl="pgfmathfunctions.code.tex"]
			640020 [style="main2  node",texlbl="pgfmathfunctions.basic.code.tex"]
			640021 [style="main2  node",texlbl="pgfmathfunctions.trigonometric.code.tex"]
			640022 [style="main2  node",texlbl="pgfmathfunctions.random.code.tex"]
			640022 [style="main2  node",texlbl="pgfmathfunctions.comparison.code.tex "]
			640023 [style="main2  node",texlbl="pgfmathfunctions.base.code.tex "]
			640024 [style="main2  node",texlbl="pgfmathfunctions.round.code.tex "]
			640025 [style="main2  node",texlbl="pgfmathfunctions.misc.code.tex "]
			640026 [style="main2  node",texlbl="pgfmathfunctions.integerarithmetics.code.tex"]
			
			
			64002->640020
			64002->640021
			64002->640022
			64002->640023
			64002->640024
			64002->640025
			64002->640026
			
		}
		\end{dot2tex}
		\end{tikzpicture}
		\caption{Beamer pgfmathfunctions.code.tex}
		\label{fig:beamer7}
	\end{figure}
\end{document}	